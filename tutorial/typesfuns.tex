\section{Language Overview}

\subsection{Primitive Types}

\Idris{} defines several primitive types: \tTC{Int}, \tTC{Integer} and
\tTC{Float} for numeric operations, \tTC{Char} and \tTC{String} for
text manipulation, and \tTC{Ptr} which represents foreign pointers.
There are also several data types declared in the library, including
\tTC{Bool}, with values \tDC{True} and \tDC{False}.
We can declare some constants with these types. Enter the following
into a file \texttt{prims.idr} and load it into the \Idris{} interactive
environment by typing \texttt{idris prims.idr}:

\begin{SaveVerbatim}{constprims}

module prims;

x : Int;
x = 42;

foo : String;
foo = "Sausage machine";

bar : Char;
bar = 'Z';

quux : Bool;
quux = False;

\end{SaveVerbatim}
\useverb{constprims}

An \Idris{} file consists of an optional module declaration (here
\texttt{module prims}) followed by an optional list of imports (none here,
however \Idris{} programs can consist of several modules, and the definitions
in each module each have their own namespace, as we will discuss shortly) and a
collection of declarations and definitions. Each definition must have a type
declaration (here, \texttt{x : Int}, \texttt{foo : String}, etc).  Each
component is separated by a semi-colon.

A library module \texttt{prelude} is automatically imported by every \Idris{} program,
including facilities for IO, arithmetic, data structures and various common
functions. The prelude defines several arithmetic and comparison operators,
which we can use at the prompt. Evaluating things at the prompt gives an
answer, and the type of the answer. For example:

\begin{SaveVerbatim}{promptprim}

*prims> 6*6+6
42 : Int
*prims> x == 6*6+6
True : Bool

\end{SaveVerbatim}
\useverb{promptprim}

All of the usual arithmetic and comparison operators are defined for the primitive
types. They are overloaded using type classes, as we will discuss in section
\ref{sec:classes} and can be extended to work on user defined types.
Boolean expressions can be tested with the \texttt{if...then...else} construct:

\begin{SaveVerbatim}{ifthenelse}

*prims> if (x==6*6+6) then "The answer!" else "Not the answer"
"The answer!" : String

\end{SaveVerbatim}
\useverb{ifthenelse}

\subsection{Data Types}



\subsection{Dependent Types}

\subsubsection{Vectors}

\subsubsection{The Finite Sets}

\subsubsection{Implicit Arguments}

\subsection{Modules and Namespaces}

