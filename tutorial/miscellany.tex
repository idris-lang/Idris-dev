\section{Miscellany}

In this section we discuss a variety of additional features: extensible syntax,
literate programming, and interfacing with external libraries through the
foriegn function interface.

\section{High Level Structures}

\subsection{\texttt{where} clauses}

\subsection{The \texttt{with} rule}

\subsection{Type Classes}

\subsection{Syntactic Sugar}



\subsection{Literate programming}

Like Haskell, \Idris{} supports \emph{literate} programming. If a file has an
extension of \texttt{.lidr} then it is assumed to be a literate file. In literate
programs, everything is assumed to be a comment unless the line begins with a
greater than sign \texttt{>}, for example:

\begin{SaveVerbatim}{litidr}

> module literate;

This is a comment. The main program is below

> main : IO ();
> main = putStrLn "Hello literate world!\n";

\end{SaveVerbatim}
\useverb{litidr}

\noindent
An additional restriction is that there must be a blank line between a program
line (beginning with \texttt{>}) and a comment line (beginning with any other
character).

\subsection{Foreign function calls}

For practical programming, it is often necessary to be able to use external libraries,
particularly for interfacing with the operating system, file system, networking, etc.
\Idris{} provides a lightweight foreign function interface for achieving this,
as part of the prelude. For this, we assume a certain amount of knowledge of
C and the \texttt{gcc} compiler. First, we define a datatype which describes the external
types we can handle:

\begin{SaveVerbatim}{foreignty}

data FTy = FInt | FFloat | FChar | FString | FPtr | FUnit;

\end{SaveVerbatim}
\useverb{foreignty}

\noindent
Each of these corresponds directly to a C type. Respectively: \texttt{int},
\texttt{float}, \texttt{char}, \texttt{char*}, \texttt{void*} and \texttt{void}.
There is also a translation to a concrete \Idris{} type, described by the
following function:

\begin{SaveVerbatim}{interpfty}

interpFTy : FTy -> Set;
interpFTy FInt    = Int;
interpFTy FFloat  = Float;
interpFTy FChar   = Char;
interpFTy FString = String;
interpFTy FPtr    = Ptr;
interpFTy FUnit   = ();

\end{SaveVerbatim}
\useverb{interpfty}

\noindent
A foreign function is described by a list of input types and a return type, which
can then be converted to an \Idris{} type:

\begin{SaveVerbatim}{ffunty}

ForeignTy : (xs:List FTy) -> (t:FTy) -> Set;

\end{SaveVerbatim}
\useverb{ffunty}

\noindent
A foreign function is assumed to be impure, so \texttt{ForeignTy} builds an
\texttt{IO} type, for example:

\begin{SaveVerbatim}{ftyex}

Idris> ForeignTy [FInt, FString] FString
Int -> String -> IO String : Set

Idris> ForeignTy [FInt, FString] FUnit 
Int -> String -> IO () : Set

\end{SaveVerbatim}
\useverb{ftyex}

\noindent
We build a call to a foreign function by giving the name of the function, a list of
argument types and the return type. The built in function \texttt{mkForeign}
converts this description to a function callable by \Idris{}

\begin{SaveVerbatim}{mkForeign}

data Foreign : Set -> Set where
    FFun : String -> (xs:List FTy) -> (t:FTy) -> 
           Foreign (ForeignTy xs t);

mkForeign : Foreign x -> x;

\end{SaveVerbatim}
\useverb{mkForeign}

\noindent
For example, the \texttt{putStr} function is implemented as follows, as a call to 
an external function \texttt{putStr} defined in the run-time system:

\begin{SaveVerbatim}{putStrex}

putStr : String -> IO ();
putStr x = mkForeign (FFun "putStr" [FString] FUnit) x;

\end{SaveVerbatim}
\useverb{putStrex}

\subsubsection*{Include and linker directives}

Foreign function calls are translated directly to calls to C functions, with appropriate
conversion between the \Idris{} representation of a value and the C representation.
Often this will require extra libraries to be linked in, or extra header and object files.
This is made possible through the following directives:

\begin{itemize}
\item \texttt{\%lib "x"} --- include the \texttt{libx} library, equivalent to passing the
\texttt{-lx} option to \texttt{gcc}.
\item \texttt{\%include "x.h"} --- use the header file \texttt{x.h}.
\item \texttt{\%obj "x.o"} --- link with the object file \texttt{x.o}.
\end{itemize}

%\subsection{Comparison}

%How does \Idris{} compare with other dependently typed languages and proof
%assistants, such as Coq, Agda and Epigram?
