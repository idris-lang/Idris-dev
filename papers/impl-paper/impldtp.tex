\documentclass{jfp1}
%\documentclass[acmtoplas]{acmtrans2m}

\usepackage[draft]{comments}
%\usepackage[final]{comments}
% \newcommand{\comment}[2]{[#1: #2]}
\newcommand{\khcomment}[1]{\comment{KH}{#1}}
\newcommand{\ebcomment}[1]{\comment{EB}{#1}}

\usepackage{epsfig}
%\usepackage{path}
\usepackage{url}
%\usepackage{amsmath,amssymb} 
\usepackage{fancyvrb}
\usepackage{todonotes}

\newenvironment{template}{\sffamily}

\usepackage{graphics,epsfig}
\usepackage{stmaryrd}

\input{./macros.ltx}
\input{./library.ltx}

\NatPackage
\FinPackage

\newcounter{per}
\setcounter{per}{1}

\newcommand{\Ivor}{\textsc{Ivor}}
\newcommand{\Idris}{\textsc{Idris}}
\newcommand{\IdrisM}{\textsc{Idris}$^-$}
\newcommand{\TT}{\textsf{TT}}
\newcommand{\TTdev}{\textsf{TT$_{dev}$}}
\newcommand{\Funl}{\textsc{Funl}}
\newcommand{\Agda}{\textsc{Agda}}
\newcommand{\LamPi}{$\lambda_\Pi$}

\newcommand{\perule}[1]{\vspace*{0.1cm}\noindent
\begin{center}
\fbox{
\begin{minipage}{7.5cm}\textbf{Rule \theper:} #1\addtocounter{per}{1}
\end{minipage}}
\end{center}
\vspace*{0.1cm}
}

\newcommand{\mysubsubsection}[1]{
\noindent
\textbf{#1}
}
\newcommand{\hdecl}[1]{\texttt{#1}}


\title
  {Implementing General Purpose Dependently Typed Programming Languages}
%\subtitle{Implementing Domain Specific Languages by Overloading}

\author[Edwin Brady]
{EDWIN BRADY\\
School of Computer Science, University of St Andrews, St Andrews,
KY16 9SX, UK}

\begin{document}

\maketitle

\begin{abstract}
Many components of a dependently typed programming language are by now well
understood, for example the underlying type theory, type checking, unification and
evaluation.  How to combine these components into a realistic and usable high
level language is, however, folklore, discovered anew by successive
language implementations.  In this paper, I describe the implementation of a
new dependently typed functional programming language, \Idris{}.
\Idris{} is intended to be a \emph{general purpose} programming language
and as such provides high level concepts such as implicit syntax, 
type classes and \texttt{do} notation. 
I describe the high level language and the underlying type theory, and present
a method for \emph{elaborating} concrete high level syntax with implicit
arguments and type classes into a fully explicit type theory. Furthermore,
I show how this method,
based on a domain specific language embedded in Haskell, facilitates the
implementation of new high level language constructs.

%I describe the implementation of a dependently typed functional
%programming language, \Idris{}. Much has been written about various
%aspects of dependently typed language implementation (e.g. checking
%dependent types, unification, optimisation) but nothing yet about how
%to bring it all together into a complete, practical, usable tool. This paper
%attempts to do so. In particular, I explain what is needed to turn 
%concrete syntax with implicit arguments into fully elaborated type
%theory, using unification and a tactic engine.
\end{abstract}


%\category{D.3.2}{Programming Languages}{Language
%  Classifications}[Applicative (functional) Languages]
%\category{D.3.4}{Programming Languages}{Processors}[Compilers]
%\terms{Languages, Verification, Performance}
%\keywords{Dependent Types, Typechecking}


%\begin{bottomstuff}
%Author's address: Edwin Brady, School of Computer Science, North Haugh, St Andrews,
%KY16 9SX
%\end{bottomstuff}

\section{Introduction}

Dependently typed programming languages, such as Agda \cite{norell2007thesis}
and Coq \cite{Bertot2004}, have emerged in recent years as a promising approach
to ensuring the correctness of software. Type checking ensures a program has
the intended meaning; \emph{dependent} types, where types may be predicated
on values, allow a programmer to give a program a more precise type and 
hence have increased confidence of its correctness.
The \Idris{} language
\cite{Brady2011a} aims to take this idea further, by providing support for
verification of \emph{general purpose} systems software. In contrast to Agda and Coq,
which have arisen from the theorem proving community, \Idris{} takes Haskell as
its main influence.  Recent Haskell extensions such as GADTs and type families have
given some of the power of dependent types to Haskell programmers. In the short
term this approach has a clear advantage, since it builds on a mature language
infrastructure with extensive library support.
Taking a longer term view, however,
these extensions are inherently limited in that they are required to maintain
backwards compatibility with existing Haskell implementations.  \Idris{}, being a new
language, has no such limitation, essentially asking the question:

\begin{center}
\emph{``What if Haskell had \emph{full} dependent types?''}
\end{center}

By \emph{full} dependent types, we mean that there is no restriction on which
values may appear in types.  It is important for the sake of usability of a
programming language to provide a notation which allows programmers to express
high level concepts in a natural way. Taking Haskell as a starting point means
that \Idris{} offers a variety of high level structures such as type 
classes, \texttt{do}-notation, primitive types and monadic I/O, for example.
Furthermore, a goal of \Idris{} is to support high level domain specific
language implementation, providing appropriate notation for \emph{domain
experts} and \emph{systems programmers}, who should not be required to be type
theorists in order to solve programming problems.  Nevertheless, it is
important for the sake of correctness of the language implementation to have a
well-defined core language with well-understood meta-theory \cite{Altenkirch2010}. 
How can we achieve both of these goals?

This paper describes a method for elaborating a high level dependently typed
functional programming language to a low level core based on dependent
type theory.  The method involves building an elaboration monad which captures
the state of incomplete programs and a collection of \emph{tactics} used by the
machine to refine programs, directed by high level syntax.  As we shall see,
this method allows higher level language constructs to be elaborated in a
straightforward manner, without compromising the simplicity of the underlying
type theory.

\subsection{Contributions}

A dependently typed programming language relies on several components, many of
which are now well understood. For example, we rely on a type checker for
the core type theory \cite{Chapman2005epigram,loh2010tutorial}, a
unification algorithm \cite{Miller1992} and an evaluator. However, it is less
well understood how to combine these components effectively into a practical
programming language. 

The primary contribution of this paper is a tactic based method for translating
programs in a high level dependently typed programming language to a small core
type theory, \TT{}, based on UTT \cite{luo1994}. The paper describes the
structure of an elaboration monad capturing proof and system state, and
introduces a collection of \remph{tactics} which the machine uses to
manipulate incomplete programs.  Secondly, the paper gives a detailed
description of the core type theory used by \Idris{}, including a full
description of the typing rules.  Finally, through describing the specific
tactics, the paper shows how to extend \Idris{} with higher level features.
While we apply these ideas to \Idris{} specifically, the method for term
construction is equally applicable to other typed programming languages, and
indeed the tactics themselves are applicable to any high level language which
can be explained in terms of \TT{}.

\subsection{Outline}

Translating an \Idris{} source program to an executable proceeds through several
phases, illustrated below:

\begin{center}
\DM{
\mbox{\Idris{}}
\;
\xrightarrow{\mathrm{ (desugaring) }}
\;
\mbox{\IdrisM{}}
\;
\xrightarrow{\mathrm{ (elaboration) }}
\;
\mbox{\TT{}}
\;
\xrightarrow{\mathrm{ (compilation) }}
\;
\mbox{Executable}
}
\end{center}

\noindent
The main focus of this paper is the elaboration phase, which translates
a desugared language \IdrisM{} into a core language \TT{}. In order to put
this into its proper context, I give an overview of the high level language
\Idris{} in Section \ref{sect:hll} and explain the core language \TT{} and
its typing rules in Section \ref{sect:typechecking}; 
Section \ref{sect:elaboration} describes the elaboration process itself,
beginning with a tactic based system for constructing
\TT{} programs, then introducing the desugared language \IdrisM{} and
showing how this is translated into \TT{} using tactics;
Section
\ref{sect:delab} describes the process for translating \TT{} back to \Idris{}
and properties of the translation; finally, 
Section \ref{sect:related} discusses related work and Section \ref{sect:conclusion}
concludes.

\subsection{Typographical conventions}

This paper presents programs in two different but related languages: a high level
language \Idris{} intended for programmers, and a low level language \TT{} to
which \Idris{} is elaborated. We distinguish these languages typographically as
follows:

\begin{itemize}
\item \Idris{} programs are written in \texttt{typewriter} font, as they are written
in a conventional text editor. We use \texttt{e$_i$} to stand for non-terminal
expressions.
\item \TT{} programs are written in mathematical notation, with names arising
from \Idris{} expressions written in \texttt{typewriter} font. We use vector notation
$\te$ to stand for sequences of expressions.
\end{itemize}

Additionally, we describe the translation from \Idris{} to \TT{} in the form
of \emph{meta-operations}, which in practice are Haskell programs. Meta-operations
are operations on \Idris{} and \TT{} syntax, and are identified by their names being
in $\MO{SmallCaps}$.

\subsection{Elaboration Example}

\Idris{} is a Haskell-like pure functional programming language with dependent
types.  A simple example program is the following, which adds corresponding
elements of vectors of the same length:

\begin{SaveVerbatim}{vadd}

vAdd : Num a => Vect n a -> Vect n a -> Vect n a
vAdd Nil       Nil       = Nil
vAdd (x :: xs) (y :: ys) = x + y :: vAdd xs ys

\end{SaveVerbatim}
\useverb{vadd}

\noindent
This illustrates some basic features of \Idris{}:

\begin{itemize}
\item Functions are defined by pattern matching, with a top level type signature.
Names which are free in the type signature (\texttt{a} and \texttt{n}) here
are implicitly bound.
\item The type system ensures that both input vectors are the same length (\texttt{n})
and have the same element type (\texttt{a}), and that the element type and length
are preserved in the output vector.
\item Functions can be overloaded using \remph{classes}. Here, the element type
of the vector \texttt{a} must be numeric and therefore supports the \texttt{+} operator.
\item A single colon is used for type signatures, and a double colon for the
cons operator, emphasising the importance of types.
\end{itemize}

\noindent
\Idris{} programs elaborate into a small core language, \TT{}, which is a $\lambda$-calculus
with dependent types, augmented with algebraic data types and pattern matching.
\TT{} programs are fully explicitly typed, including the names \texttt{a} and \texttt{n}
in the type signature, and the names bound in each pattern match clause.
Classes are also made explicit.
The \TT{} translation of \texttt{vAdd} is:

\DM{
\AR{
\FN{vAdd} \Hab(\va\Hab\Type)\to(\vn\Hab\Nat)\to\TC{Num}\;\va\to
  \Vect\;\vn\;\va\to\Vect\;\vn\;\va\to\Vect\;\vn\;\va\\
\RW{var}\;\va\Hab\Type,\;\vc\Hab\TC{Num}\;\va\SC\\
\hg\FN{vAdd} \; \va \; \Z \; \vc \; (\DC{Nil}\;\va) \; (\DC{Nil}\;\va) \; 
\cq\;\DC{Nil}\;\va \\
\RW{var}\;\AR{
\va\Hab\Type,\;\vk\Hab\Nat,\;\vc\Hab\TC{Num}\;\va,\\
\vx\Hab\va,\;\vxs\Hab\Vect\;\vk\;\va,\;
\vy\Hab\va,\;\vys\Hab\Vect\;\vk\;\va
\SC
\\
\FN{vAdd} \; \va \; (\suc\;\vk) \; \vc 
  \; ((\DC{::})\;\va\;\vk\;\vx\;\vxs) 
  \; ((\DC{::})\;\va\;\vk\;\vy\;\vys) 
   \\
   \hg\hg\cq\:((\DC{::})\;\va\;\vk\;((+)\;\vc\;\vx\;\vy)\; (\FN{vAdd}\;\va\;\vk\;\vc\;\vxs\;\vys))
}
}
}

\noindent
The rest of this paper describes \Idris{} and \TT{}, and systematically explains
how to translate from one to the other.



%\input{overview.tex}

\section{\Idris{} --- the High Level Language}

\label{sect:hll}

\Idris{} is
a pure functional programming language with dependent types. It is
eagerly evaluated by default, and compiled via the Epic supercombinator
library~\cite{brady2011epic}, with irrelevant values and proof terms
automatically erased~\cite{Brady2003,Brady2005}.
In this section, I present some small examples to illustrate
the features of \Idris{}.
A full tutorial is available elsewhere \cite{idristutorial}. 

\subsection{Preliminaries}

\Idris{} defines several primitive types: fixed width integers
\tTC{Int}, arbitrary width integers \tTC{Integer},
\tTC{Float64} for numeric operations, \tTC{Char} and \tTC{String} for
text manipulation, and \tTC{Ptr} which represents foreign pointers.
There are also several data types declared in the library, including
\tTC{Bool}, with values \tDC{True} and \tDC{False}. All of the usual
arithmetic and comparison operators are defined for the primitive types.

An \Idris{} program consists of a module declaration, followed by an optional
list of imports and a collection of definitions and declarations, for example:

\begin{SaveVerbatim}{constprims}

module Main

x : Int
x = 42

main : IO ()
main = putStrLn ("The answer is " ++ show x)

\end{SaveVerbatim}
\useverb{constprims}

\noindent
Like Haskell, the main function is called \texttt{main}, and input and output
is managed with an \texttt{IO} monad. Unlike Haskell, however, \remph{all} 
top-level functions must have a type signature. This is due to type inference
being, in general, undecidable for languages with dependent types.

A module declaration also opens a \remph{namespace}. The fully qualified names
declared in this module are \texttt{Main.x} and \texttt{Main.main}.

\subsection{Data Types}

Data types may be declared in a similar way to Haskell data types, with a
similar syntax. For example, Peano natural numbers and lists
are respectively declared in the library as follows:

\begin{SaveVerbatim}{natdecl}

data Nat    = Z   | S Nat 
data List a = Nil | (::) a (List a)

\end{SaveVerbatim}
\useverb{natdecl}

\noindent
A standard example of a \remph{dependent} type is the type of ``lists with
length'', conventionally called ``vectors'' in the dependently-typed
programming literature. In \Idris{}, vectors are declared as follows:

\begin{SaveVerbatim}{vectdecl}

data Vect : Nat -> Type -> Type where
   Nil  : Vect Z a
   (::) : a -> Vect k a -> Vect (S k) a

\end{SaveVerbatim}
\useverb{vectdecl}

\noindent
The above declaration creates a family of types. The syntax resembles a Haskell
GADT declaration: it explicitly states the type
of the type constructor \tTC{Vect} --- it takes a \tTC{Nat} and a type as
arguments, where \tTC{Type} stands for the type of types. We say that \tTC{Vect}
is \emph{indexed} over \tTC{Nat} and \emph{parameterised} by a type. 
The distinction between parameters and indices is that a parameter is fixed
across an entire data structure, whereas an index may vary.
Note that constructor names may be overloaded (such as \texttt{Nil} and
\texttt{(::)} here) provided that they are declared in separate 
namespaces.

Note that in the library, \texttt{List} is declared in a module
\texttt{Prelude.List}, and \texttt{Vect} in a module \texttt{Prelude.Vect}, so
the fully qualified names of \texttt{Nil} are \texttt{Prelude.List.Nil} and
\texttt{Prelude.Vect.Nil}. They may be used qualified or unqualified; if
used unqualified the elaborator will disambiguate by type.

\subsection{Functions}

Functions are implemented by pattern matching. For example, the following
function defines concatenation of vectors, expressing in the type that
the resulting vector's length is the sum of the input lengths:

\begin{SaveVerbatim}{vapp}

(++) : Vect n a -> Vect m a -> Vect (n + m) a
Nil       ++ ys = ys
(x :: xs) ++ ys = x :: xs ++ ys

\end{SaveVerbatim}
\useverb{vapp}

\noindent
Functions can also be defined \emph{locally} using \texttt{where} clauses. For
example, to define a function which reverses a list, we can use an auxiliary
function which accumulates the new, reversed list, and which does not need to
be visible globally:

\begin{SaveVerbatim}{revwhere}

reverse : List a -> List a
reverse xs = revAcc Nil xs where
  revAcc : List a -> List a -> List a
  revAcc acc Nil = acc
  revAcc acc (x :: xs) = revAcc (x :: acc) xs

\end{SaveVerbatim}
\useverb{revwhere}

\subsubsection*{Totality checking}

Internally, \Idris{} programs are checked for \emph{totality} --- that they
produce an answer in finite time for all possible inputs --- but they are not
\emph{required} to be total by default. Totality checking serves two
purposes: firstly, if a program terminates for all inputs 
then its type gives a strong guarantee about the properties specified by its
type; secondly, we can optimise total programs more
aggressively~\cite{Brady2003}. 

Totality checking can be enforced by using the \texttt{total} keyword.
For example, the following definition is accepted:

\begin{SaveVerbatim}{vaddt}

total vAdd : Num a => Vect n a -> Vect n a -> Vect n a
vAdd Nil       Nil       = Nil
vAdd (x :: xs) (y :: ys) = x + y :: vAdd xs ys

\end{SaveVerbatim}
\useverb{vaddt}

\noindent
The elaborator can verify that this is total by checking that it covers all
possible patterns --- in this case, both arguments must be of the same form
as the type requires that the input vectors are the same length --- and that
recursive calls are on structurally smaller values. 

\Idris{} also supports coinductive definitions, though this and further
details of totality checking are beyond the scope of this paper; the totality
checker is implemented independently of elaboration and type checking.

\subsection{Implicit Arguments}

In order to understand how \Idris{} elaborates to an underlying type theory,
it is important to understand the notion of \remph{implicit} arguments.
To illustrate this, consider the finite sets:

\begin{SaveVerbatim}{findecl}

data Fin : Nat -> Type where
      fZ : Fin (S k)
      fS : Fin k -> Fin (S k)

\end{SaveVerbatim}
\useverb{findecl}

\noindent
As the name suggests, these are sets with a finite number of elements.
The declaration gives
\tDC{fZ} as the zeroth element of a finite set with \texttt{S k} elements,
and \texttt{fS n} as the
\texttt{n+1}th element of a finite set with \texttt{S k} elements. 

\tTC{Fin} is indexed by \tTC{Nat}, which represents the number of elements in
the set.  Neither constructor targets \texttt{Fin Z}, because the empty
set has no elements. The following function uses an element
of a finite set as a bounds-safe index into a vector:

\begin{SaveVerbatim}{vindex}

index : Fin n -> Vect n a -> a
index fZ     (x :: xs) = x
index (fS k) (x :: xs) = index k xs

\end{SaveVerbatim}
\useverb{vindex}

\noindent
Let us take a closer look at its type.
It takes two arguments, an element of the finite set of \texttt{n} elements, and a vector
with \texttt{n} elements of type \texttt{a}. But there are also two names, 
\texttt{n} and \texttt{a}, which are not declared explicitly. These are \emph{implicit}
arguments to \texttt{index}. The type of \texttt{index} could also be written as:

\begin{SaveVerbatim}{vindeximppl}

index : {a:_} -> {n:_} -> Fin n -> Vect n a -> a

\end{SaveVerbatim}
\useverb{vindeximppl}

\noindent
This gives bindings for \texttt{a} and \texttt{n} with placeholders for
their types, to be inferred by the machine. These types could also be given explicitly:

\begin{SaveVerbatim}{vindeximpty}

index : {a:Type} -> {n:Nat} -> Fin n -> Vect n a -> a

\end{SaveVerbatim}
\useverb{vindeximpty}

\noindent
Implicit arguments, given in braces \texttt{\{\}} in the type signature, are
not given in applications of \texttt{index}; their values can be inferred from
the types of the \texttt{Fin n} and \texttt{Vect n a} arguments. Any name which
appears in a non-function position in a type signature, but which is otherwise
free, will be automatically bound as an implicit argument.  Implicit arguments
can still be given explicitly in applications, using the syntax
\texttt{\{a=value\}} and \texttt{\{n=value\}}, for example:

\begin{SaveVerbatim}{vindexexp}

index {a=Int} {n=2} fZ (2 :: 3 :: Nil)

\end{SaveVerbatim}
\useverb{vindexexp}

\subsection{Classes}

\Idris{} supports overloading in two ways. Firstly, as we have already seen
with the constructors of \texttt{List} and \texttt{Vect}, names can be
overloaded in an ad-hoc manner and resolved according to the context in which
they are used. This is mostly for convenience, to eliminate the need to
decorate constructor names in similarly structured data types, and eliminate
explicit qualification of ambiguous names where only one is well-typed --- this
is especially useful for disambiguating record field names, although records
are beyond the scope of this paper.

Secondly, \Idris{} implements \remph{classes}, following Haskell's type
classes.  This allows a more principled approach to overloading --- a class
gives a collection of overloaded operations which describe the interface for
\remph{instances} of that class. \Idris{} classes follow Haskell~98 type
classes, except that multiple parameters are supported, and that classes can be
parameterised by \remph{any} value, not just types.
Hence we refer to
them as \remph{classes} generally, rather than \remph{type classes}
specifically.

A simple example is the \texttt{Show} class, which is defined in the
library and provides an interface for converting values to \texttt{String}s:

\begin{SaveVerbatim}{showclass}

class Show a where
    show : a -> String

\end{SaveVerbatim}
\useverb{showclass}

\noindent
An instance of a class is defined with an \texttt{instance} declaration, which
provides implementations of the function for a specific type. For example:

\begin{SaveVerbatim}{shownat}

instance Show Nat where
    show Z = "Z"
    show (S k) = "s" ++ show k

\end{SaveVerbatim}
\useverb{shownat}

\noindent
Instance declarations can themselves have constraints. For example, to define a
\texttt{Show} instance for vectors, we need to know that there is a
\texttt{Show} instance for the element type, because we are going to use it to
convert each element to a \texttt{String}:

\begin{SaveVerbatim}{showvec}

instance Show a => Show (Vect n a) where
    show xs = "[" ++ show' xs ++ "]" where
        show' : Vect n a -> String
        show' Nil        = ""
        show' (x :: Nil) = show x
        show' (x :: xs)  = show x ++ ", " ++ show' xs

\end{SaveVerbatim}
\useverb{showvec}

\noindent
Classes can also be extended. A logical next step from an equality relation
\texttt{Eq} for example is to define an ordering relation \texttt{Ord}. We can
define an \texttt{Ord} class which inherits methods from \texttt{Eq} as well as
defining some of its own:

\begin{SaveVerbatim}{ord}

data Ordering = LT | EQ | GT

\end{SaveVerbatim}
\useverb{ord} 

\begin{SaveVerbatim}{eqord}

class Eq a => Ord a where
    compare : a -> a -> Ordering
    (<) : a -> a -> Bool
    -- etc

\end{SaveVerbatim}
\useverb{eqord}

\subsection{Matching on intermediate values}

%\subsubsection{\texttt{let} bindings}
%
%Intermediate values can be calculated using \texttt{let} bindings:
%
%\begin{SaveVerbatim}{letb}
%
%mirror : List a -> List a
%mirror xs = let xs' = rev xs in
%                app xs xs'
%
%\end{SaveVerbatim}
%\useverb{letb} 
%
%\noindent
%Pattern matching is also supported in \texttt{let} bindings. For example, extracting
%fields from a record can be achieved as follows, as well as by pattern matching at the top level:
%
%\begin{SaveVerbatim}{letp}
%
%data Person = MkPerson String Int
%
%showPerson : Person -> String
%showPerson p = let MkPerson name age = p in
%                   name ++ " is " ++ show age ++ " years old"
%
%\end{SaveVerbatim}
%\useverb{letp} 

\subsubsection{\texttt{case} expressions}

Intermediate values of \emph{non-dependent} types can be inspected using a
\texttt{case} expression.  For example, \texttt{list\_lookup} looks up an index
in a list, returning \texttt{Nothing} if the index is out of bounds. This can
be used to write \texttt{lookup\_default}, which looks up an index and
returns a default value if the index is out of bounds:

\begin{SaveVerbatim}{listlookup}

lookup_default : Nat -> List a -> a -> a
lookup_default i xs def = case list_lookup i xs of
                              Nothing => def
                              Just x => x

\end{SaveVerbatim}
\useverb{listlookup} 

The \texttt{case} construct is intended for simple analysis of intermediate
expressions to avoid the need to write auxiliary functions.  It will
only work if each branch \emph{matches} a value of the same type, and
\emph{returns} a value of the same type.

\subsubsection{The \texttt{with} rule}

Often, matching is required on the result of an intermediate computation
with a dependent type.
\Idris{} provides a construct for this, the \texttt{with} rule, 
inspired by views in \Epigram~\cite{McBride2004a},
which takes account of the
fact that matching on a value in a dependently-typed language can affect what
is known about the forms of other values. 
%
For example, a \texttt{Nat} is either even or odd. 
If it is even it will
be the sum of two equal \texttt{Nat}s. Otherwise, it is the sum of two equal \texttt{Nat}s 
plus one:

\begin{SaveVerbatim}{parity}

data Parity : Nat -> Type where
   even : Parity (n + n)
   odd  : Parity (S (n + n))

\end{SaveVerbatim}
\useverb{parity}

\noindent
We say \texttt{Parity} is a \emph{view} of \texttt{Nat}. 
It has a \emph{covering function} which tests whether
it is even or odd and constructs the predicate accordingly:

\begin{SaveVerbatim}{parityty}

parity : (n:Nat) -> Parity n

\end{SaveVerbatim}
\useverb{parityty}

\noindent
Using this, a function which converts a natural number to a list
of binary digits (least significant first) is written as follows, using the \texttt{with}
rule:

\begin{SaveVerbatim}{natToBin}

natToBin : Nat -> List Bool
natToBin Z = Nil
natToBin k with (parity k)
   natToBin (j + j)     | even = False :: natToBin j
   natToBin (S (j + j)) | odd  = True  :: natToBin j

\end{SaveVerbatim}
\useverb{natToBin}

\noindent
The value of the result of \texttt{parity k} affects the form of \texttt{k}, 
because the result
of \texttt{parity k} depends on \texttt{k}. 
So, as well as the patterns for the result of the
intermediate computation (\texttt{even} and \texttt{odd}) right of the 
\texttt{$\mid$}, the definition also expresses how
the results affect the other patterns left of the $\mid$. Note that there is a
function in the patterns (\texttt{+}) and repeated occurrences of \texttt{j} --- 
this is allowed
because another argument has determined the form of these patterns. In general,
non-linear patterns are allowed \emph{only} when their value is forced by
some other argument.




\input{typechecking.tex}

%\input{proofstate.tex}

\input{elaboration.tex}

\input{delab.tex}

%\input{compiling.tex}

%\section{Syntax Extensions}

\Idris{} supports the implementation of Embedded Domain Specific Languages (EDSLs) in
several ways~\cite{res-dsl-padl12}. One way, as we have already seen, is through
extending \texttt{do} notation. Another important way is to allow extension of the core
syntax. In this section we describe two ways of extending the syntax: \texttt{syntax}
rules and \texttt{dsl} notation.

\subsection{\texttt{syntax} rules}

We have seen \texttt{if...then...else} expressions, but these
are not built in --- instead, we define a function in the prelude\ldots

\begin{SaveVerbatim}{boolelim}

boolElim : (x:Bool) -> |(t : a) -> |(f : a) -> a; 
boolElim True  t e = t;
boolElim False t e = e;

\end{SaveVerbatim}
\useverb{boolelim}

\noindent
\ldots and extend the core syntax with a \texttt{syntax} declaration:

\begin{SaveVerbatim}{syntaxif}

syntax if [test] then [t] else [e] = boolElim test t e;

\end{SaveVerbatim}
\useverb{syntaxif}

\noindent
The left hand side of a \texttt{syntax} declaration describes the syntax rule, and the right
hand side describes its expansion. The syntax rule itself consists of:

\begin{itemize}
\item \textbf{Keywords} --- here, \texttt{if}, \texttt{then} and \texttt{else}, which must
be valid identifiers
\item \textbf{Non-terminals} --- included in square brackets, \texttt{[test]}, \texttt{[t]}
and \texttt{[e]} here, which stand for arbitrary expressions. To avoid parsing ambiguities, 
these expressions cannot use syntax extensions at the top level (though they can be used
in parentheses).
\item \textbf{Names} --- included in braces, which stand for names which may be bound
on the right hand side.
\item \textbf{Symbols} --- included in quotations marks, e.g. \texttt{":="}. This can
also be used to include reserved words in syntax rules, such as \texttt{"let"} or \texttt{"in"}.
\end{itemize}

\noindent
The limitations on the form of a syntax rule are that it must include at least
one symbol or keyword, and there must be no repeated variables standing for
non-terminals. Any expression can be used, but if there are two non-terminals
in a row in a rule, only simple expressions may be used (that is, variables,
constants, or bracketed expressions). Rules can use previously defined rules,
but may not be recursive.  The following syntax extensions would therefore be
valid:

\begin{SaveVerbatim}{syntaxex}

syntax [var] ":=" [val]              = Assign var val;
syntax [test] "?" [t] ":" [e]        = if test then t else e;
syntax select [x] from [t] where [w] = SelectWhere x t w;
syntax select [x] from [t]           = Select x t;

\end{SaveVerbatim}
\useverb{syntaxex}

\noindent
Syntax macros can be further restricted to apply only in patterns (i.e., only on the left
hand side of a pattern match clause) or only in terms (i.e. everywhere but the left hand side
of a pattern match clause) by being marked as \texttt{pattern} or \texttt{term} syntax
rules. For example, we might define an interval as follows, with a static check
that the lower bound is below the upper bound using \texttt{so}:

\begin{SaveVerbatim}{interval}

data Interval : Type where
   MkInterval : (lower : Float) -> (upper : Float) -> 
                so (lower < upper) -> Interval

\end{SaveVerbatim}
\useverb{interval}

\noindent
We can define a syntax which, in patterns, always matches \texttt{oh} for the proof 
argument, and in terms requires a proof term to be provided:

\begin{SaveVerbatim}{intervalsyn}

pattern syntax "[" [x] "..." [y] "]" = MkInterval x y oh
term    syntax "[" [x] "..." [y] "]" = MkInterval x y ?bounds_lemma

\end{SaveVerbatim}
\useverb{intervalsyn} 

\noindent
In terms, the syntax \texttt{[x...y]} will generate a proof obligation
\texttt{bounds\_lemma} (possibly renamed).

Finally, syntax rules may be used to introduce alternative binding forms. For
exampe, a \texttt{for} loop binds a variable on each iteration:

\begin{SaveVerbatim}{forloop}

syntax for {x} "in" [xs] ":" [body] = forLoop xs (\x => body)
  
main : UnsafeIO ()
main = do for x in [1..10]:
              putStrLn ("Number " ++ show x)
          putStrLn "Done!"

\end{SaveVerbatim}
\useverb{forloop} 

\noindent
Note that we have used the \texttt{\{x\}} form to state that \texttt{x} represents
a bound variable, substituted on the right hand side. We have also put \texttt{"in"} in
quotation marks since it is already a reserved word.

\input{dsl}



\section{Related Work}

\label{sect:related}

Dependently typed programming languages have become more prominent in recent
years as tools for verifying software correctness, and several experimental
languages are being developed, in particular Agda \cite{norell2007thesis},
Epigram \cite{McBride2004a,Levitation2010} and Trellys \cite{Kimmell2012}.
Furthermore, recent extensions to Haskell~\cite{Vytiniotis2011}, implemented in the
Glasgow Haskell Compiler, are bringing more of the power of dependent types to
Haskell. The problem of refining high level syntax to a core type theory also
applies to theorem provers based on dependent types such as
Coq~\cite{Bertot2004} and Matita~\cite{Asperti2011}.

Checking advanced type system features in Haskell involves a type system
parameterised over an underlying constraint system $X$ which captures
constraints such as type classes, constrained data types and type families.
Types are checked using an inference algorithm
\textsc{OutsideIn(X)}~\cite{Vytiniotis2011}, which is stratified into an
inference engine independent of the constraint
system, and a constraint solver for $X$. An additional difficulty faced by
Haskell, and hence any extensions, is the desire to support type \emph{inference}, in
which principal types may be inferred for top level functions. We have avoided
such difficulties since, in general, type
inference is undecidable for full dependent types. Indeed, it is not clear
that type inference is even desirable in many cases, as programmers
can use dependent types to state their intentions (hence a program
specification) more precisely. However in future work we
may consider adapting the \textsc{OutsideIn} approach to provide limited type
inference.

An earlier implementation of \Idris{} was built on the \Ivor{} proof engine
\cite{Brady2006b}. This implementation differed in one important way --- unlike
the present implementation, there was limited separation between the type
theory and the high level language. The type theory itself supported implicit
syntax and unification, with high level constructs such as the \texttt{with}
rule implemented directly. Two important disadvantages were found with this
approach, however: firstly, the type checker is much more complicated when
combined with unification, making it harder to maintain; secondly, adding new
high level features requires the type checker to support those features
directly. In contrast, elaboration by tactics gives a clean separation between
the low level and high level languages, and results in programs in a core
type theory which is separately checkable, perhaps even by an independently
written checker which implements the \TT{} rules directly.

Matita uses a bi-directional refinement algorithm~\cite{Asperti}. This is a
type directed approach, maintaining a set of yet to be solved unification
problems and relying on a small kernel, similar to the approach we now take
with \Idris{}, However, their approach uses refinement rules rather than
tactics. This leads to good error messages though it is not clear how
easy it would be to extend to additional high level language features, unlike
the tactic based approach.

The Agda implementation is based on a type theory with
implicit syntax and pattern matching --- Norell gives an algorithm for type checking
a dependently typed language with pattern matching and metavariables 
\cite{norell2007thesis}. Unlike the present \Idris{} implementation, metavariables
and implicit arguments are part of the type theory. This has the advantage that
implicit arguments can be used more freely (for example, in higher order
function arguments) at the expense of complicating the type system.

Epigram \cite{McBride2004a} and Oleg \cite{McBride1999} 
have provided much of the inspiration for the \Idris{} elaborator. Indeed,
the hole and guess bindings of \TTdev{} are taken directly from Oleg.
\Epigram{} does not implement pattern matching directly, but rather translates
pattern matching into elimination rules \cite{McBride2002}. This has the
advantage that
elimination rules provide termination and coverage proofs \emph{by construction}.
Furthermore they simplify implementation of the evaluator and provide easy
optimisation opportunities \cite{Brady2003}. However, it requires the
implementation of extra machinery for constructor manipulation
\cite{McBride2006} and so we have avoided it in the present implementation.



%[Observation: separate elaboration and type checking, sort of like in GHC which
%type checks the high level language and produces a type correct core language.
%Elaboration is effectively a type checker for the high level language, so we have
%a hope of providing reasonable error messages related to the original code.]

\section{Conclusion}

\label{sect:conclusion}

In this paper, I have given an overview of the programming language \Idris{},
and its core type theory \TT{}, giving a detailed algorithm for translating
high level programs into \TT{}.
\TT{} itself is deliberately small and simple, and the design has deliberately
resisted innovation so that we can rely on existing metatheoretic properties
being preserved. The kernel of the \Idris{} implementation consists of a type checker
and evaluator for \TT{} along with a pattern match compiler, which are implemented
in under 1000 lines of Haskell code. It is important that this kernel remains small
--- the correctness of the language implementation relies to a large extent on
the correctness of the underlying type system, and keeping the implementation small
reduces the possibility of errors.

The approach we have taken to implementing the high level language,
implementing an elaboration monad with tactics for program construction, allows
us to build programs on top of a small and unchanging kernel, rather than
extending the core language to deal with implicit syntax, unification and type
classes.  High level \Idris{} features are implemented by describing the
corresponding sequence of tactics to build an equivalent program in \TT{}, via
a development calculus of incomplete terms, \TTdev{}. A significant advantage
we have found with this approach is that higher level features can easily be
implemented in terms of existing elaborator components. For example, once we
have implemented elaboration for data types and functions, it is easy to add
several features:

\begin{itemize}
\item \textbf{Type classes}: A dictionary is merely a record containing the
functions which implement a type class instance. Since we have a tactic based
refinement engine, we can implement type class resolution as a tactic.
\item \textbf{\texttt{where} clauses}: We have access to local variables and
their types, so we can
elaborate \texttt{where} clauses at the point of definition simply by lifting
them to the top level. 
\item \textbf{\texttt{case} expressions}: Similar to \texttt{where} clauses,
these are implemented by lifting the branches out to a top level function.
\end{itemize}

We do not need to make any changes to the core language type system in order to 
implement these high level features. 
Other high level features such as dependent records, tuples and monad comprehensions
can be added equally easily --- and indeed have been added in the full implementation.  
Furthermore, 
since we have taken a tactic-based approach to elaborating \Idris{} to \TT{},
it is possible to expose tactics to the programmer. This opens up
the possibility of implementing domain specific decision procedures, or
implementing user defined tactics in a style similar to Coq's \texttt{Ltac}
language \cite{Delahaye2000}.  Although \TT{} is primarily intended as a core
language for \Idris{}, its rich type system also means that it could be used
as a core language for other high level languages, especially when augmented
with primitive operators, and used to express additional properties of those
languages.  

We have not discussed the performance of the elaboration
algorithm, or described how \Idris{} compiles to executable code. In practice,
we have found performance to be acceptable --- for example, the \Idris{}
library (31 files, 3718 lines of code in total at the time of writing)
elaborates in around 12 seconds\footnote{On a MacBook Pro, 2.8GHz Intel Core 2
Duo, 4Gb RAM}. Profiling suggests that the main bottleneck is locating holes
in a proof term, which can be improved by choosing a better representation
for proof terms, perhaps based on a zipper \cite{Huet1997}. Compilation is made
straightforward by the Epic library \cite{brady2011epic}, with I/O and foreign
functions handled using command-response interaction trees \cite{Hancock2000}.
Although I have not yet run detailed benchmarks of the performance of compiled
code, I believe that rich type information and guaranteed termination will
allow aggressive optimisations~\cite{Brady2003,Brady2005}. I will investigate
this in future work.

The objective of this implementation of \Idris{} is to provide a platform
for experimenting with realistic, general purpose programming with dependent
types, by implementing a Haskell-like language augmented with \emph{full}
dependent types. 
In this paper, we have seen how such a high level language can be implemented
by building on top of a small, well-understood, easy to reason about type
theory. 
However, a programming language implementation is not an end in itself. 
Programming languages exist to support research and practice in many different
domains. In future work, therefore, I plan to apply domain specific
language based techniques to realistic problems in important safety critical
domains such as security and network protocol design and implementation. In
order to be successful, this will require a language which is expressive enough
to describe protocol specifications at a high level, and robust enough to
guarantee correct implementation of those protocols. \Idris{}, I believe, is
the right tool for this work.



\section*{Acknowledgements}

This work was funded by the Scottish Informatics and Computer Science
Alliance (SICSA) and by EU Framework 7 Project No. 248828 (ADVANCE).
My thanks to Kevin Hammond and Vilhelm Sj\"{o}berg for their comments 
on an earlier draft of this paper.

\bibliographystyle{jfp}
\bibliography{library.bib}

\appendix

%\section{TODOs}
%\listoftodos{}

%\section{Elaboration meta-operations}

%It's possible that it would be useful to have a quick reference of meta-operations
%used by the elaborator here.

%They are: $\ttinterp{\cdot}$, $\MO{Elab}$, $\MO{TTDecl}$,
%$\MO{NewProof}$, $\MO{NewTerm}$,
%$\MO{Term}$, $\MO{Type}$, $\MO{Context}$, $\MO{Patterns}$, $\MO{Lift}$, $\MO{Expand}$.

%\input{code}

\end{document}
