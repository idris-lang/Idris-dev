\newcommand{\ttinterp}[1]{\mathcal{E}\interp{#1}}

\section{Elaborating \Idris{}}

An \Idris{} program consists of a series of declarations --- data types, functions,
type classes and instances. In this section, we describe how these high level declarations
are translated into a \TT{} program consisting of inductive families and pattern matching
function definitions. We will need to work at the \remph{declaration} level, and at
the \remph{expression} level, defining the following meta-operations:

\begin{itemize}
\item $\ttinterp{\cdot}$, which builds a \TT{} expression from an \Idris{} expression
\item $\MO{Elab}$, which processes a top level \Idris{} declaration by generating
one or more \TT{} declarations.
\end{itemize}

\subsection{The Development Calculus \TTdev}

We build \TT{} expressions by using high level \Idris{} expressions to
direct a tactic based theorem prover, which builds the \TT{} expressions
step by step, by refinement. In order to build expressions in this way,
the type theory needs to support
\remph{incomplete} terms, and a method for term construction. 
To achieve this, we extend \TT{} with \remph{holes},
calling the extended calculus \TTdev{}.
Holes stand for the parts of programs which have not yet been
instantiated; this largely follows the \Oleg{} development
calculus~\cite{McBride1999}.

The basic idea is to extend the syntax for binders with a \remph{hole}
binding and a \remph{guess} binding. 
These extensions are given in Figure \ref{ttdev}.
The \remph{guess} binding is
similar to a $\LET$ binding, but without any computational force,
i.e. the bound names do not reduce.
Using binders to represent holes is useful in a dependently typed setting since
one value may determine another. Attaching a ``guess'' to a binder ensures that
instantiating one such value also instantiates all of its dependencies. The
typing rules for binders ensure that no $?$ bindings leak into types.

\FFIG{
\AR{
\vb ::= \ldots 
 \:\mid\: \hole{\vx}{\vt} \:\:(\mbox{hole binding}) \:\:
 \:\mid\: \guess{\vx}{\vt}{\vt} \:\:(\mbox{guess})
\medskip\\
\Rule{
\Gamma;\hole{\vx}{\vS}\proves\ve\Hab\vT
}
{
\Gamma\proves\hole{\vx}{\vS}\SC\ve\Hab\vT
}
\hspace*{0.1cm}\vx\not\in\vT
\hspace*{0.1in}\mathsf{Hole}
\hg
\Rule{
\Gamma;\guess{\vx}{\vS}{\ve_1}\proves\ve_2\Hab\vT
}
{
\Gamma\proves\guess{\vx}{\vS}{\ve_1}\SC\ve_2\Hab\vT
}
\hspace*{0.1cm}\vx\not\in\vT
\hspace*{0.1in}\mathsf{Guess}
}
}
{\TTdev{} extensions}
{ttdev}


\subsection{Proof State}

A proof state is a tuple, $(\vC, \Delta, \vT, \vQ)$, containing:

\begin{itemize}
\item A global context, $\vC$, containing pattern matching definitions
\item A local context, $\Delta$, containing pattern bindings
\item A proof term, $\vT$, in \TTdev{}
\item A hole queue, $\vQ$
%\item \remph{Deferred} definitions, $\vD$, for introducing global metavariables
\end{itemize}

The \remph{hole queue} is a list of names of hole and guess binders in the proof term ---
we ensure that each bound name is unique. Holes essentially refer to \remph{sub goals}
in the proof.
When this queue is empty, the proof term is complete.
Creating a \TT{} expression from an \Idris{} expresson involves creating
a new proof state, with an empty proof term, and using the high level definition
to direct the building of a final proof state, with a complete proof term.

In the implementation, the proof state is captured in an elaboration monad,
\texttt{Elab}, which includes various operations for querying and updating
the proof state, manipulating terms, generating fresh names, etc. However, we will
describe \Idris{} elaboration in terms of meta-operations on the proof state,
in order to capture the essence of the elaboration process without being distracted
by implementation details. These meta-operations include: 

\begin{itemize}
\item \demph{Queries} which retrieve values from the proof state, without modifying
the state. For example, we can:
\begin{itemize}
\item Retrieve the local context $\Gamma$ at the current sub goal
\item Type check or normalise a term relative to $\Gamma$
\item Unify two terms (potentially solving sub goals) relative to $\Gamma$ 
\end{itemize}
\item \demph{Tactics} which update the proof term. Tactics operate on the sub term
at the binder specified by the head of the hole queue $\vQ$.
\item \demph{Focussing} on a specific sub goal, which brings a different sub goal to the
head of the hole queue.
\item \demph{Deferring} a sub goal, which adds a new definition to the global context
$\vC$ which solves the sub goal.
\end{itemize}

Elaboration of an \Idris{} expression involves creating a new proof state, running
a series of tactics to build a complete proof term, then retrieving and \remph{rechecking}
the final proof term, which must be a \TT{} program (i.e. does not contain any of the
\TTdev{} extensions).

\subsection{System State}

We need the global state, for implicits, global context.

\subsection{Elaborating Expressions}


\IdrisM{}, a subset of \Idris{} not including syntactic sugar (e.g. pairs, do notation, etc).

Implicit and type class arguments? Expanded at the application site (we need to know
it's the global name after all and we do that by type).

\subsection{Elaborating Data Types}

\subsection{Elaborating Pattern Matching}

