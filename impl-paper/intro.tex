\section{Introduction}

Dependently typed programming languages, such as Agda \cite{norell2007thesis}
and Coq \cite{Bertot2004}, have emerged in recent years as a promising approach
to ensuring the correctness of software. The \Idris{} language
\cite{Brady2011a} aims to take these ideas further, by providing support for
verification of general purpose systems software. In contrast to Agda and Coq,
which have arisen from the theorem proving community, \Idris{} takes Haskell as
its main influence.  Recent Haskell extensions such as GADTs and type families have
given some of the power of dependent types to Haskell programmers. In the short
term this approach has a clear advantage, since it builds on a mature language
infrastructure with extensive library support.
Taking a longer term view, however,
these extensions are inherently limited in that they are required to maintain
backwards compatibility with existing Haskell implementations.  \Idris{}, being a new
language, has no such limitation, essentially asking the question:

\begin{center}
\emph{``What if Haskell had \emph{full} dependent types?''}
\end{center}

It is important for the sake of usability of a programming language to provide
a notation which allows programmers to express high level concepts
in a natural way. Taking Haskell as a starting point means that \Idris{} offers
high level structures such as type classes, \texttt{do}-notation, primitive types
and monadic I/O, for example.
Nevertheless, it is important for the sake of correctness of the language implementation
to have a small core language with well understood meta-theory
\cite{Altenkirch2010}.  
How can we achieve both of these goals?

This paper describes a method for elaborating a high level dependently typed
functional programming language to a low level core language based on dependent
type theory.  The method involves building an Embedded Domain Specific Language
(EDSL) for constructing programs in the core language, using \emph{tactics}
directed by high level program syntax.  
As we shall see, this method allows higher level language constructs to be
constructed in a straightforward manner, without compromising the simplicity
of the underlying type theory.

\subsection{Overview}

A dependently typed programming language relies on several components, many of
which are now well understood. For example, we rely on a type checker for
the core type theory \cite{Chapman2005epigram,loh2010tutorial}, a
unification algorithm \cite{Miller1992} and an evaluator. However, it is less
well understood how to combine these components effectively into a practical
programming language. Therefore, the primary contribution of this paper is
an EDSL based method for translating programs in a
high level dependently typed programming language to a small core type theory,
\TT{}, based on UTT \cite{luo1994}. The paper describes the structure of the
EDSL including proof and system state, and introduces a collection of \remph{tactics}
for manipulating incomplete programs.
Secondly, the paper gives a detailed description of the core type theory used
by \Idris{}, including a full description of the typing rules. Finally, through
describing the EDSL and the specific tactics, the paper shows how to extend
\Idris{} with higher level features.
While we apply these ideas to \Idris{} specifically, the method for term construction
is equally applicable to other typed programming languages.

The paper is structured as follows: In Section \ref{sect:hll} I give an overview of
programming in \Idris{}, introducing the high level constructs which will be
translated to the core language; Section \ref{sect:typechecking} describes the
core language \TT{} and its typing rules, covering expressions, data types
and pattern matching; Section \ref{sect:elaboration} describes the method
for translating \Idris{} to \TT{}, beginning with a restricted language
\IdrisM{} before extending to higher level features; finally, in 
Section \ref{sect:discussion} we discuss related work and conclude.


\subsection{Typographical conventions}

This paper presents programs in two different but related languages: a high level
language \Idris{} intended for programmers, and a low level language \TT{} to
which \Idris{} is elaborated. We distinguish these languages typographically as
follows:

\begin{itemize}
\item \Idris{} programs are written in \texttt{truetype}, as they are written
in a conventional text editor. We use \texttt{e$_i$} to stand for non-terminal
expressions.
\item \TT{} programs are written in mathematical notation, with names arising
from \Idris{} expressions written in \texttt{truetype}. We use vector notation
$\te$ to stand for sequences of expressions.
\end{itemize}

Additionally, we describe the translation from \Idris{} to \TT{} in the form
of \emph{meta-operations}, which in practice are Haskell programs. Meta-operations
are operations on \Idris{} and \TT{} syntax, and are identified by their names being
in $\MO{SmallCaps}$.


