\section{Reversing Elaboration}

\label{sect:delab}

As well as translating from \Idris{} to \TT{}, so that programs can be type
checked and evaluated, it is valuable to define the reverse transformation. This
serves two principal purposes:

\begin{itemize}
\item To assist the user, it is preferable that the results of evaluation, and any
error messages produced by the elaborator, are presented in \Idris{} syntax
rather than \TT{}.
\item For correctness, we would like to ensure as far as possible that the 
result of elaboration is equivalent to the original program. Informally, we can
achieve this by checking that reversing the elaboration process yields the original
program (with implicit arguments expanded).
\end{itemize}

\noindent
In this section, we describe the process for reversing elaboration and the required
properties of the elaboration process as a whole. Fortunately, translating from
\TT{} to \Idris{} is significantly easier than \Idris{} to \TT{}, because it is
primarily \emph{erasing} information.

\subsection{From \TT{} to \Idris{}}

\subsection{Elaboration Properties}

\todo[inline]{
What are the properties of elaboration?
Need to define unelaboration, and say that elaborating then unelaborating
an expression yields the original expression.
(Works for expressions but not declarations)
}

Properties:
\begin{itemize}
\item Elaboration produces a well-typed term
\item $\ve \MO{Matches} \uninterp{\ttinterp{\ve}}$
\end{itemize}


